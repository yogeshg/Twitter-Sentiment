\subsection{Datasets}
One of the major challenges in Sentiment Analysis of Twitter is to collect a
labelled dataset. Researchers have made public the following datasets for
training and testing classifiers.

\subsubsection{Twitter Sentiment Corpus} This is a collection of 5513 tweets
collected for four different topics, namely, Apple, Google, Microsoft, Twitter
It is collected and hand-classified by Sanders Analytics LLC \cite{San}. Each
entry in the corpus contains, Tweet id, Topic and a Sentiment label. We use
Twitter-Python library to enrich this data by downloading data like Tweet text,
Creation Date, Creator etc. for every Tweet id. Each Tweet is
hand classified by an American male into the following four categories. For the
purpose of our experiments, we consider Irrelevant and Neutral to be the same
class. Illustration of Tweets in this corpus is show in \tabref{tab:TSC}.
%TODO \cite{TwPyLib}

\begin{description}
	\item[Positive] {For showing positive sentiment towards the topic}
	\item[Positive] {For showing no or mixed or weak sentiments towards the topic}
	\item[Negative] {For showing negative sentiment towards the topic}	
	\item[Irrelevant] {For non English text or off-topic comments}
\end{description}

\begin{table}[h!]
\centering

\begin{tabular}{|l|r|p{0.75\textwidth}|}
\hline
Class & Count & Example \\\hline
neg & 529 & {{\#}Skype often crashing: {\#}microsoft, what are you doing?} \\\hline
neu & 3770 & {How {\#}Google Ventures Chooses Which Startups Get Its \$200
				Million http://t.co/FCWXoUd8 via @mashbusiness @mashable} \\\hline
pos & 483 & {Now all @Apple has to do is get swype on the iphone and
				it will be crack. Iphone that is} \\\hline

\end{tabular}

\caption{Twitter Sentiment Corpus}
\label{tab:TSC}
\end{table}

\subsubsection{Stanford Twitter} This corpus of tweets, developed by Sanford’s
Natural Language processing research group, is publically available \cite{GBH}.
The training set is collected by querying Twitter API for happy emoticons like
``\texttt{:)}'' and sad emoticons like ``\texttt{:(}'' and labelling them
positive or negative. The emoticons were then stripped and Re-Tweets and
duplicates removed. It also contains around 500 tweets manually collected and
labelled for testing purposes. We randomly sample and use 5000 tweets from this
dataset. An example of Tweets in this corpus are shown in \tabref{tab:STAN}.

\begin{table}[h!]
\centering

\begin{tabular}{|l|r|p{0.75\textwidth}|}
\hline
Class & Count & Example \\\hline
neg & 2501 & Playing after the others thanks to TV scheduling may well allow us to know what's go on, but it makes things look bad on Saturday nights  \\\hline
pos & 2499 & @francescazurlo HAHA!!! how long have you been singing that song now? It has to be at least a day. i think you're wildly entertaining!  \\\hline

\end{tabular}

\caption{Stanford Corpus}
\label{tab:STAN}
\end{table}
