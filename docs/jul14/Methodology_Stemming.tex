\subsection{Stemming Algorithms} All stemming algorithms are of the following
major types – affix removing, statistical and mixed. The first kind, Affix
removal stemmer, is the most basic one. These apply a set of transformation
rules to each word in an attempt to cut off commonly known prefixes and / or
suffixes \cite{Ilia}. A trivial stemming algorithm would be to truncate words at N-th
symbol. But this obviously is not well suited for practical purposes.

J.B. Lovins described first stemming algorithm in 1968. It defines 294 endings,
each linked to one of 29 conditions, plus 35 transformation rules\ignore{need to write
more, conditions, transformations?}. For a word being stemmed, an ending with a
satisfying condition is found and removed. Another famous stemmer used
extensively is described in the next section.

\subsubsection{Porter Stemmer}

Martin Porter wrote a stemmer that was published in July 1980. This stemmer was
very widely used and became and remains the de facto standard algorithm used for
English stemming. It offers excellent trade-off between speed, readability, and
accuracy. It uses a set of around 60 rules applied in 6 successive steps \cite{Porter}. An
important feature to note is that it doesn’t involve recursion. The steps in the
algorithm are described in \tabref{tab:porter}.

\begin{table}[h!]
\centering
\begin{tabular}{|r|l|} \hline
1.	&	Gets rid of plurals and -ed or -ing suffixes \\ \hline
2.	&	Turns terminal y to i when there is another vowel in the stem \\ \hline
3.	&	Maps double suffixes to single ones: -ization, -ational, etc. \\ \hline
4.	&	Deals with suffixes, -full, -ness etc. \\ \hline
5.	&	Takes off -ant, -ence, etc. \\ \hline
6.	&	Removes a final –e \\ \hline
\end{tabular}
\caption{Porter Stemmer Steps}
\label{tab:porter}
\end{table}

\ignore{Another commonly used algorithm is [blah]. Which does [blah].}

\subsubsection{Lemmatization}

Lemmatization is the process of normalizing a word rather than just finding its
stem. In the process, a suffix may not only be removed, but may also be
substituted with a different one. It may also involve first determining the
part-of-speech for a word and then applying normalization rules. It might also
involve dictionary look-up. For example, verb ‘saw’ would be lemmatized to ‘see’
and the noun ‘saw’ will remain ‘saw’. For our purpose of classifying text,
stemming should suffice.
