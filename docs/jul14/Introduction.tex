\section{Introduction}

\subsection{Sentiment Analysis} Sentiment Analysis refers to the use of text
analysis and natural language processing to identify and extract subjective
information in textual contents. There are two type of user-generated content
available on the web – facts and opinions. Facts are statements about topics and
in the current scenario, easily collectible from the Internet using search
engines that index documents based on topic keywords. Opinions are user specific
statement exhibiting positive or negative sentiments about a certain topic.
Generally opinions are hard to categorize using keywords. Various text analysis
and machine learning techniques are used to mine opinions from a document
\cite{survey}. Sentiment Analysis finds its application in a variety of domains.

\begin{description}

\item[Business]{ Businesses may use sentiment analysis on blogs, review websites
etc. to judge the market response of a product. This information may also be
used for intelligent placement of advertisements. For example, if product ``A'' and
``B'' are competitors and an online merchant business ``M'' sells both, then ``M'' may
advertise for ``A'' if the user displays positive sentiments towards ``A'', its brand or
related products, or ``B'' if they display negative sentiments towards ``A''. }

\item[Government]{ Governments and politicians can actively monitor public
sentiments as a response to their current policies, speeches made during
campaigns etc. This will help them make create better public awareness regarding
policies and even drive campaigns intelligently. }

\item[Financial Markets]{ Public opinion regarding companies can be used to
predict performance of their stocks in the financial markets. If people have a
positive opinion about a product that a company A has launched, then the share
prices of A are likely to go higher and vice versa. Public opinion can be used
as an additional feature in existing models that try to predict market
performances based on historical data. }

\end{description}

\subsection{Twitter} Twitter is an online social networking and micro-blogging
service that enables users to create and read short messages, called ``Tweets''.
It is a global forum with the presence of eminent personalities from the
field of entertainment, industry and politics. People tweet about their life,
events and express opinion about various topics using text messages limited to
140 characters. Registered users can read and post tweets, but any unregistered
users can read them. Twitter can be accessed via Web, SMS, or mobile apps.
Traditionally a large volume of research in sentiment analysis and opinion
mining has been directed towards larger pieces of text like movie reviews.
Sentiment Analysis in micro-blogging sphere is relatively new. From the
perspective of Sentiment Analysis, we discuss a few characteristics of Twitter:

\begin{description}

\item[Length of a Tweet]{ The maximum length of a Twitter message is 140
characters. This means that we can practically consider a tweet to be a single
sentence, void of complex grammatical constructs. This is a vast difference from
traditional subjects of Sentiment Analysis, such as movie reviews. }

\item[Language used]{ Twitter is used via a variety of media including SMS and
mobile phone apps. Because of this and the 140-character limit, language used in
Tweets tend be more colloquial, and filled with slang and misspellings.
\ignore{, as compared to other user-generated content on the web.} Use of
hashtags also gained popularity on Twitter and is a primary feature in any given
tweet. Our analysis shows that there are approximately 1-2 hashtags per tweet, as
shown in \tabref{tab:feat_freq}. }

\item[Data availability]{ Another difference is the magnitude of data available.
With the Twitter API, it is easy to collect millions of tweets for training. There
also exist a few datasets that have automatically and manually labelled the tweets
\cite{GBH} \cite{San}. }

\item[Domain of topics]{ People often post about their likes and dislikes on social
media. These are not al concentrated around one topic. This makes twitter a unique
place to model a generic classifier as opposed to domain specific classifiers that
could be build datasets such as movie reviews. }

\end{description}
